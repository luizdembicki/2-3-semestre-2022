\documentclass[12pt, a4paper, twoside, chapter=TITLE, subsection=TITLE, section=TITLE, subsubsection=TITLE, subsubsubsection=TITLE, english, german, brazil]{abntex2}


\usepackage[utf8]{inputenc}
\usepackage[brazilian,hyperpageref]{backref}
\usepackage{pacotes}


\titulo{Processo de produção de metanol}
\tituloestrangeiro{--texto do título estrangeiro--}
\autor{
    Angelo \\
    Isabelli \\
    Luiz \\
}
\orientador{--Professor--}
\coorientador{--Monitor--}
\data{2022}
\instituicao{
    Universidade Federal do Paraná (UFPR) \par
    Departamento de Engenharia Química \par
    Graduação em Engenharia Química
}
\tipotrabalho{Trabalho Acadêmico}
\local{Curitiba, PR}
\preambulo{Trabalho acadêmico para a disciplina de Integração de Processos 1}

% informações do PDF
\makeatletter
\hypersetup{
    pdftitle={\@title},
    pdfauthor={\@author},
    pdfsubject={\imprimirpreambulo},
    pdfkeywords={PALAVRAS}{CHAVE}{EM}{PORTUGUES},
    pdfcreator={LaTeX with abnTeX2},
    colorlinks=true,
    linkcolor=blue,
    citecolor=blue,
    urlcolor=blue
    }
\makeatother
\setlength{\absparsep}{18pt} % ajusta o espaçamento dos parágrafos do resumo

\begin{document}
\pretextual 
\imprimircapa
\imprimirfolhaderosto[]* % talvez dê merda na numeração das páginas
\makeatother\cleardoublepage


\begin{resumo}[Resumo] 
    ------- Resumo em português. -----  \\
    \vspace{\onelineskip}
    \noindent
    \textbf{Palavras-chave}: latex. abntex. editoração de texto.
\end{resumo}

\begin{resumo}[Abstract]
    \begin{otherlanguage*}{english}
        ------ abstract. ---- \\
        \vspace{\onelineskip}
        \noindent
        \textbf{Keywords}: latex. abntex. publication de textes.
    \end{otherlanguage*}
\end{resumo}


% \pdfbookmark[0]{\listfigurename}{lof}
% \listoffigures*
% \cleardoublepage


% \pdfbookmark[0]{\listtablename}{lot}
% \listoftables*
% \cleardoublepage


% \begin{siglas}
%   \item[sigla] significado da sigla;
% \end{siglas}


% \begin{simbolos}
%     \item[simbolo] descrição;
% \end{simbolos}


\pdfbookmark[0]{\contentsname}{toc}
\tableofcontents*
\cleardoublepage


\textual % a partir daqui são os elementos textuais.
% \include{XXX} % adicionar os .tex da pasta inputs

\postextual % a partir daqui são os elementos pós-textuais.
\bibliography{ref} % O alerta amarelo ("Empty `thebibliography' environment on input line 8.") é porque não há citações ainda
% \chapter{APÊNDICE}
% \chapter{anexo}
\end{document}
