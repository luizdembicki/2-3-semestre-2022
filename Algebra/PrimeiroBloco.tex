\documentclass[12pt, twoside, a4]{article} % fonte 12, frente e verso, a4

\usepackage{setspace}
\usepackage[fleqn,tbtags]{mathtools}
\usepackage{amsmath}
\usepackage{amssymb}

\begin{document}

\title{Primeiro bloco de exércicios - Algebra Linear}
\author{Luiz Augusto Dembicki Fernandes}
\date{03/12/2022}
\maketitle

\section{Provar que a associatividade da adição de vetores é teorema em $ R^2 $ usual.}

\quad    Definindo que $ u \coloneqq (a, b)$, $v  \coloneqq  (c, d)$ e $w  \coloneqq  (e, f)$, partimos de:
$ (u + v) + w $ , realizamos as somas pela definição de adição de vetores em R²:
\begin{center}
\begin{doublespacing}
    
Aplicando a definição \\ 
$ (u + v) + w = ((a, b) + (c, d)) + (e, f)$ \\
Somando $u + v$ e em seguida $ (u + v) + w$ \\
$ \to ((a + c), (b + d)) + (e, f) =  ((a + c) + e, (b + d) + f)$ \\
Utilizando-se que nos reais a adição é associativa \\
$ \to ((a + c) + e, (b + d) + f) = ((c + e) + a, (d + f) + b)$ \\
Que por definição de adição de matrizes \\
$ \to  ((c + e) + a, (d + f) + b) = ((c + e), (d + f)) + (a, b)$ \\
$ \therefore ((c + e), (d + f)) + (a, b) = ((c, d) + (e, f)) + (a, b)$, Por adição de vetores, \\
Que é igual ao ponto de partida \\
$ ((c, d) + (e, f)) + (a, b) =  ((a + c), (b + d)) + (e, f)$ \\
Usando a definição dos vetores \\
\end{doublespacing}
$ \therefore (u + v) + w = u + (v + w) $

\end{center}

\section{Escolher um axioma qualquer de espaço vetorial real e provar que ele é teorema em P² usual.}

\end{document}