\documentclass[12pt, twoside, a4]{article} % fonte 12, frente e verso, a4

\usepackage{setspace}
\usepackage[fleqn,tbtags]{mathtools}
\usepackage{amsmath}
\usepackage{amssymb}
\usepackage{geometry}
\geometry{a4paper, left=35mm, right=35mm, top=51mm, bottom=30mm}

\begin{document}

\title{Primeiro bloco de exércicios - Álgebra Linear}
\author{Luiz Augusto Dembicki Fernandes}
\date{03/12/2022}
\maketitle

\section{Provar que a associatividade da adição de vetores é teorema em $ R^2 $ usual.}

\quad    Definindo que $ u \coloneqq (a, b)$, $v  \coloneqq  (c, d)$ e $w  \coloneqq  (e, f)$, partimos de:
$ (u + v) + w $ , realizamos as somas pela definição de adição de vetores em R²:
\begin{center}
    \begin{doublespacing}
        \vspace{-1cm}
        Aplicando a definição \\ 
        $ (u + v) + w = ((a, b) + (c, d)) + (e, f)$ \\
        Somando $u + v$ e em seguida $ (u + v) + w$ \\
        $ \to ((a + c), (b + d)) + (e, f) =  ((a + c) + e, (b + d) + f)$ \\
        Utilizando-se que nos reais a adição é associativa \\
        $ \to ((a + c) + e, (b + d) + f) = ((c + e) + a, (d + f) + b)$ \\
        Que por definição de adição de matrizes \\
        $ \to  ((c + e) + a, (d + f) + b) = ((c + e), (d + f)) + (a, b)$ \\
        $ \therefore ((c + e), (d + f)) + (a, b) = ((c, d) + (e, f)) + (a, b)$, Por adição de vetores, \\
        Que é igual ao ponto de partida \\
        $ ((c, d) + (e, f)) + (a, b) =  ((a + c), (b + d)) + (e, f)$ \\
        Usando a definição dos vetores \\
    \end{doublespacing}
    $ \therefore (u + v) + w = u + (v + w) $
\end{center}

\section{Escolher um axioma qualquer de espaço \\ vetorial real e provar que ele é teorema em $P_2$ usual.}

\begin{large}
    \begin{bf}
        V8: Se $u$ pertence a $V$ , então existe $v$ pertencente a $V$ tal que
        $u + v = \overline{0}$;
    \end{bf} 
\end{large} \\ \\
Seja $\mathcal{P}_2 :$
$\langle P_2, \mathbb{R}, +, \cdot, \bigcirc \rangle $ \\ \\
Sendo $p$ e $q$ funções polinomiais reais de grau menor ou igual a 2, $+$ é então definido como $(p + q)(x) = p(x) + q(x)$. \\ \\
$\bigcirc$ é a função $ g(x) = 0 ~ | ~ \forall x \in \mathbb{R} $. \\
\begin{center}
    \vspace{-2cm}
    Definindo: \\  $p(x) = \alpha x^2 + \beta x + \delta ~ | ~ \alpha \in \mathbb{R}, \beta \in \mathbb{R}, \delta \in \mathbb{R}$ \\
   e $q(x) = \pi x^2 + \varpi  x + \xi  ~ | ~ \pi \in \mathbb{R}, \varpi  \in \mathbb{R}, \xi  \in \mathbb{R}$

    \begin{doublespacing}

        Devido a definição de adição em $\mathcal{P}_2 :$ \\
        $(p + q)(x) = p(x) + q(x)$, Substituindo $p(x)$ e $q(x)$ \\

       $p(x) + q(x) = \alpha x^2 + \beta x + \delta + \pi x^2 + \varpi  x + \xi $ \\
        Por distributividade da multiplicação de reais
        $\to \alpha x^2 + \beta x + \delta + \pi x^2 + \varpi  x + \xi = (\alpha + \pi)x^2 + (\beta + \varpi)x + (\delta + \xi)$ \\
        Como reais, existem inversos aditivos para cada elemento, possibilitando as seguintes expressões:
        $ \alpha + \pi = 0 , \beta + \varpi = 0, \delta + \xi = 0$ \\
        $\to \alpha = -\pi , \beta = - \varpi, \delta = - \xi$ \\
        Que então para este caso particular \\
        $\to (\alpha + \pi)x^2 + (\beta + \varpi)x + (\delta + \xi) = (0)x^2 + (0)x + (0)$ \\
        Assim para todo x real a função retornará 0, o que equivale ao vetor nulo do espaço: 
        $(0)x^2 + (0)x + (0) = \bigcirc$ \\ 
        Portanto, tais que $\alpha = -\pi , \beta = - \varpi, \delta = - \xi$ \\
        Existe  $p(x) + q(x) = \bigcirc$;
    \end{doublespacing}
\end{center}

\section{Provar que qualquer reta que passa pela origem de $R^2$ define subespaço de $R^2$ usual.}

Utilizando o espaço vetorial $ \mathbb{R}^2 $  usual:
\begin{center}
    $ \mathcal{V} = \langle \mathbb{R}^2, \mathbb{R}, +, \cdot , (0,0)    \rangle  $ \\
    Onde         
    \begin{doublespacing}
        $ + : \mathbb{R}^2 \times \mathbb{R}^2 \to \mathbb{R}^2  $ ,é uma função tal que $ +((a, b), (c, d)) = (a, b) + (c, d) = (a + c, b + d)$  \\
        $ \cdot : \mathbb{R} \times \mathbb{R}^2 \to \mathbb{R}^2 $ ,é uma função tal que $ \cdot (\alpha, (a, b)) = \alpha \cdot (a, b) = (\alpha a, \alpha b) $
    \end{doublespacing}
\end{center}
\
Definindo: 

\begin{center}
    \begin{doublespacing} \vspace{-0.8cm}
        $ \mathcal{R} = \langle r, \mathbb{R},\oplus , \odot , (0,0) \rangle $
        \\
        Tal que $ r = \{  (x, y) \in \mathbb{R}^2 \ | $ \\ $\ ax + by = 0 \wedge a \in \mathbb{R} \wedge b \in \mathbb{R} \wedge ( ( a = 0 \to b \neq 0 ) \vee (b = 0 \to a \neq 0) ) \}  $\footnote{\emph{após ax + by = 0, a condição é que a e b são reais e não podem ser simultaneamente nulos}}
        \\
        Onde $ \oplus : r \times r \to \mathbb{R}^2 $ é restrição de +
        \\
        e $ \odot : \mathbb{R} \times r \to \mathbb{R}^2$ é restrição de $\cdot$
    \end{doublespacing}
\end{center}
\noindent
Sejam elementos de $r$ $: (m, n)$ e $ (p, q) $ \\
Então por definição:
\vspace{-0.5cm}
\begin{center}
    \begin{doublespacing}
        $ am + bn = 0$ 
        \\
        $ ap + bq = 0 $ 
        \\
        Por igualdade
        \\
        $\Rightarrow am + bn + ap + bq = 0$ 
        \\
        Por distributividade da multiplicação de reais
        \\
        $\Leftrightarrow  a(m + p) + b(n + q) = 0 $
        \\
        Assim se encaixa na definição de $r$ a soma dos elementos
        \\
        $\Rightarrow  (m+p, n + q) \in r$
        \\
        E por definição de $ \oplus $
        \\
        $(m+p) \oplus (n+q) = (m+p, n+q) $
        \\
        Assim a operação $ \oplus $ é fechada em $ r $, e por definição de $\odot$
        \\
        $ \alpha \odot (p,q) = (\alpha p , \alpha q) $
        \\
        Como o elemento acima pertence a $r$ por definição de $\odot$
        \\
        $\Rightarrow a \alpha p +b \alpha q = 0$
        \\
        Assim o elemento pertence a $r$
        \\
        $ \Rightarrow (\alpha p , \alpha q) \in r $
        \\
        O elemento $(0,0)$ respeita a definição
        \\
        $a(0) + b(0) = 0$
        \\
        E portanto pertence a $r$
        \\
        $(0,0) \in r$

    \end{doublespacing}
\end{center}

Assim $ r \subseteq \mathbb{R}^2 $, e ambos contém o mesmo vetor nulo e as funções da quintupla ordenada de $\mathcal{R} $ são restrições respectivas de $ \mathbb{R}^2 $ usual, 
assim $ \mathcal{R} $ é um subespaço de $ \mathbb{R}^2 $.


\section{$ \langle \mathbb{R}^3, \mathbb{R}, +, \cdot, (0, 0, 0)\rangle  $ Usual} 
\begin{itemize} 
    \begin{bf}
        \item Dar exemplo de um conjunto x de vetores que gera o espaço mas é L.D.
        \item Dar exemplo de conjunto y de vetores L.I. que não gera o espaço.
    \end{bf}
\end{itemize}
Utilizando as definições:
\begin{center}

        
        $\sum\limits_{n = 1}^{n} \alpha_i \cdot v_i = \bar{0}$  
        \\
        É linearmente independente se, e somente se, a equação acima admite solução única, onde $\alpha_i$ são escalares, $v_i $ são vetores de um conjunto pertencente ao espaço vetorial em análise, e $\bar{0}$ é o vetor nulo de tal interpretação.  
        \vspace{0.5cm}
        \\
        Um Conjunto de vetores de um mesmo espaço vetorial gera tal espaço vetorial em análise se, e somente se, todo vetor pertencente a tal possa ser obtido por combinação linear de tal conjunto.
        \\

\end{center}
\noindent
Seja $x = \{(1,0,0), (0,1,0), (0,0,1), (6,6,6)\}$
\vspace{-0.3cm}
\begin{center}
    \begin{doublespacing}
        Por definição
        \\
        $ \alpha_1 \cdot (1,0,0) +  \alpha_2 \cdot (0,1,0) +\alpha_3 \cdot (0,0,1) + \alpha_4  \cdot (6,6,6) = (0,0,0)$
        \\
        Existem mais do que uma solução para o acima:
        \\
        $ \alpha_1=6 ; \alpha_2 = 6 ;\alpha_3 =6 ; \alpha_4 = -1$
        \\
        $ \alpha_1=-6 ; \alpha_2 = -6 ;\alpha_3 =-6 ; \alpha_4 = 1$
        \\
        Assim o conjunto é linearmente dependente
        \\
        Para a condição de geração do espaço vetorial, um vetor qualquer de $\mathbb{R}^3$ usual
        \\
        (a, b, c)
        \\
        $ \alpha_1=(a - 6) ; \alpha_2 = (b - 6) ;\alpha_3 =(c - 6) ; \alpha_4 = 1$
        \\ 
        $ \alpha_1 \cdot (1,0,0) +  \alpha_2 \cdot (0,1,0) +\alpha_3 \cdot (0,0,1) + \alpha_4  \cdot (6,6,6) = (a,b,c)$
        \\
        Substituindo e efetuando as multiplicações entre escalares e vetores
        \\
        $ (a - 6) \cdot (1,0,0) +  (b - 6) \cdot (0,1,0) + (c - 6) \cdot (0,0,1) + 1\cdot (6,6,6) =  ((a - 6),0,0) +   (0,(b - 6),0) +  (0,0,(c - 6)) + (6,6,6)$
        \\
        Realizando a soma entre vetores
        \\
        $\to  ((a - 6) + 6,(b - 6) + 6,(c - 6) + 6)$
        \\
        Por associatividade de reais
        \\
        $\to  ((6 - 6) + a,(6 - 6) + b,(6 - 6) + c) =  ((0) + a,(0) + b,(0) + c)$
        \\
        $\Rightarrow (a, b, c) = (a, b, c) $

    \end{doublespacing}
\end{center}
\noindent
Seja $y = \{ (1,0,0), (0,1,0) \} $
\vspace{-0.3cm}
\begin{center}
    \begin{doublespacing}
        Por definição
        \\
        $ \alpha_1 \cdot (1,0,0) +  \alpha_2 \cdot (0,1,0) = (0,0,0)$
        \\
        Existe uma única solução para o acima:
        \\
        Por multiplicação de escalares reais por vetores reais
        \\
        $ \alpha_1 \cdot (1,0,0) +  \alpha_2 \cdot (0,1,0) = (\alpha_1 \cdot 1, \alpha_2 \cdot 1,0)$
        \\
        Já que 1 é neutro multiplicativo
        \\
        $\to (\alpha_1, \alpha_2,0) = (0,0,0) $
        \\
        A equação acima implica que $\alpha_1 = 0$ e $ \alpha_2 = 0$, a única solução real
        \\
        Assim o conjunto é linearmente independente
        \\
        Para a condição de geração do espaço vetorial, um vetor qualquer de $\mathbb{R}^3$ usual
        \\
        (a, b, c)
        \\ 
        $ \alpha_1 \cdot (1,0,0) +  \alpha_2 \cdot (0,1,0) = (a,b,c)$
        \\
        Efetuando as multiplicações entre escalares reais e vetores reais, e como 1 é neutro multiplicativo
        \\
        $\to  (\alpha_1 1,0,0) +  (0,\alpha_2 1,0) = (\alpha_1,0,0) +  (0,\alpha_2,0) $
        \\
        Realizando a soma entre vetores
        \\
        $\to (\alpha_1,0,0) +  (0,\alpha_2,0) = (\alpha_1,\alpha_2,0)$
        \\
        Para a igualdade abaixo
        \\
        $(\alpha_1,\alpha_2,0) = (a,b,c)$
        \\
        Implica que $c=0$, qual em $\mathbb{R}^3$ assume qualquer valor $\mathbb{R}$, que acarreta que o conjunto $y$ não gera $\mathbb{R}^3$.

    \end{doublespacing}
\end{center}

\end{document}