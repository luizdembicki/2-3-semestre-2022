\documentclass[12pt, twoside, a4]{article} % fonte 12, frente e verso, a4

\usepackage{setspace}
\usepackage[fleqn,tbtags]{mathtools}
\usepackage{amsmath}
\usepackage{amssymb}
\usepackage{geometry}
\geometry{a4paper, left=35mm, right=35mm, top=51mm, bottom=30mm}

\begin{document}

\title{Segundo bloco de exércicios - Álgebra Linear}
\author{Luiz Augusto Dembicki Fernandes}
\date{07/02/2023}
\maketitle

\section{Provar que a imagem de uma transformada linear qualquer é subespaço do contradomínio}

\section{Dados espaços vetoriais:}
\begin{doublespacing}
    \begin{large}
         \begin{bf}

            $\langle V, \mathbb{R}, +_V, \cdot_V, \overrightarrow{0}_V  \rangle $ e 
           $\langle W, \mathbb{R}, +_W, \cdot_W, \overrightarrow{0}_W  \rangle $ \\
           Transformação de $ T: V \to W \ \ | \ \ v \mapsto T(v) $ \\
           Provar que a $ Im(T) $ é subsespaço de $\langle W, \mathbb{R}, +_W, \cdot_W, \overrightarrow{0}_W  \rangle $
        
        \end{bf}
    \end{large}
\end{doublespacing}



\section{$\langle M_{3x2}, \mathbb{R}, +, \cdot, 
\begin{pmatrix} 
 0 & 0 \\
 0 & 0 
\end{pmatrix} \rangle 
$ Exibir um produto interno e dois vetores ortogonais entre si neste produto interno}


\end{document}