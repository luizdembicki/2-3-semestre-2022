\documentclass[12pt, twoside, a4]{article} % fonte 12, frente e verso, a4

\usepackage{setspace}
\usepackage[fleqn,tbtags]{mathtools}
\usepackage{amsmath}
\usepackage{amssymb}
\usepackage{geometry}
\geometry{a4paper, left=35mm, right=35mm, top=51mm, bottom=30mm}

\begin{document}

\title{Segundo bloco de exércicios - Álgebra Linear}
\author{Luiz Augusto Dembicki Fernandes}
\date{19/02/2023}
\maketitle

\section{Dados espaços vetoriais:}
\vspace{-0.3cm}
\begin{Large}
        \begin{bf}
        $\langle V, \mathbb{R}, +_V, \cdot_V, \overrightarrow{0}_V  \rangle $ e 
        $\langle W, \mathbb{R}, +_W, \cdot_W, \overrightarrow{0}_W  \rangle $ \\
        Transformação de $ \mathcal{T} : V \to W \ \ | \ \ v \mapsto \mathcal{T}(v) $ \\
        Provar que a $ \mathfrak{Im} (\mathcal{T} ) $ é subsespaço de $\langle W, \mathbb{R}, +_W, \cdot_W, \overrightarrow{0}_W  \rangle $  
    \end{bf}
\end{Large}

A definição de imagem da transformada:
\[ \mathfrak{Im} (\mathcal{T} ) = \{w \in W | \exists v(v \in V \wedge \mathcal{T} (v) = w)\} \]

O espaço advindo da transformada em questão:
\[ \langle \mathfrak{Im} (\mathcal{T}),\ \mathbb{R},\ \oplus ,\ \odot, \bigcirc \rangle \]

Para que seja subsespaço de $\langle W, \mathbb{R}, +_W, \cdot_W, \overrightarrow{0}_W  \rangle $ :
\begin{enumerate}
    \item $\mathfrak{Im} (\mathcal{T}) \subseteq W $
    
    Por definição de $\mathfrak{Im} (\mathcal{T}) $:
    \[ \mathfrak{Im} (\mathcal{T} ) = \{w \in W | \exists v(v \in V \wedge \mathcal{T} (v) = w)\} \implies \mathfrak{Im} (\mathcal{T}) \subseteq W \]
    \item $ \bigcirc \in W $
    
    Partindo do teorema de que um vetor multiplicado pelo escalar nulo é igual ao vetor nulo do mesmo espaço:
    \[ 0 \cdot_V v = \overrightarrow{0}_V  \label{eq: eq1} \tag{1} \]
    Portanto:
    \[  \mathcal{T}(\overrightarrow{0}_V ) = \mathcal{T}(0 \cdot_V v)\]
    E da definição de transformação linear: 
    \[ \to \mathcal{T}(0 \cdot_V v) = 0 \cdot_W \mathcal{T}(v)\]
    Então utilizando de \ref{eq: eq1}, no espaço vetorial de $W$:
    \[ 0 \cdot_W \mathcal{T}(v) =  \overrightarrow{0}_W \implies \bigcirc = \overrightarrow{0}_W \] 
    \[ \implies \bigcirc \in W \]
    \item $ \oplus $ é restrição de $ +_W $
    
    Definindo:
    \[ w_1 \in \mathfrak{Im}(\mathcal{T}) \wedge w_2 \in \mathfrak{Im}(\mathcal{T}) \]
    \[ v_1 \in V \wedge w_1 \in W \ | \ \mathcal{T}(v_1) = w_1 \]
    \[ v_2 \in V \wedge w_2 \in W \ | \ \mathcal{T}(v_2) = w_2 \]
    Então, utilizando da definição de transformação linear:
    \[ \mathcal{T}( v_1 +_V v_2 ) = \mathcal{T}( v_1 ) +_W \mathcal{T}( v_2) = w_1 +_W w_2 \]
    Assim  $ \oplus $ é restrição de $ +_W $

    \item $ \odot  $  é restrição de $ \cdot_W $
    
    Dado um escalar $ \alpha \in \mathbb{R} $, e os vetores explicitados no item anterior.
    Usando a definição de transformação linear:
    \[  \mathcal{T}(  \alpha \cdot_V v_1 ) = \alpha \cdot_W  \mathcal{T}( v_1 ) = \alpha \cdot_W w_1\]
    Portanto $ \odot  $  é restrição de $ \cdot_W $
    
\end{enumerate}

Com isso $ \langle \mathfrak{Im} (\mathcal{T}),\ \mathbb{R},\ \oplus ,\ \odot, \bigcirc \rangle $ é subsespaço de $\langle W, \mathbb{R}, +_W, \cdot_W, \overrightarrow{0}_W  \rangle $

\section{Seja $\langle C^0 ([2,-2]),\ \mathbb{R},\ +, \cdot, \ 0  \rangle $
 um espaço vetorial real munido de produto interno: }
\begin{Large}
        \begin{bf}
        \[ \langle f, g  \rangle = \int_{-2}^{2} f(x)g(x)dx \]  
        Demonstre o cálculo de distância entre as funções $ f(x) = 2x $ e $ g(x) = 1 $,
        ou seja $d(2x,\ 1)$,
        pertencentes ao espaço vetorial descrito anteriormente.  

    \end{bf}
\end{Large}
\

Definindo a função distância($ d $):
    \[ d (u, v) = \Vert u - v \Vert = \sqrt[]{\langle (u - v), (u - v) \rangle} \]
Utilizando as funções desejadas na definição:
    \[ d(f, g) = \sqrt[]{\langle (f - g), (f - g) \rangle} =  \sqrt[]{\langle (2x - 1), (2x - 1) \rangle}\]
Aplicando o produto interno:
    \[ d(f, g) = \sqrt[]{\int_{-2}^{2} (2x - 1)(2x - 1)dx} = \sqrt[]{\int_{-2}^{2} 4x^2 -4x + 1 \ dx} \]
Resolvendo a integral:   
    \[ d(f, g) = \sqrt[]{ \frac{4 x^3}{3} -2x^2 + x \ \Big\vert_{-2}^{2}} \]
Avaliando os limites de integração:
    \[ d(f, g) = \sqrt{ \frac{76}{3} } = \frac{2}{3} \sqrt{57} \]

\section{Em $\langle M_{3 \times 2}, \mathbb{R}, +, \cdot,
\begin{psmallmatrix} 
 0 & 0 \\
 0 & 0 \\
 0 & 0
\end{psmallmatrix} \rangle 
$ exibir um produto interno e dois vetores ortogonais entre si para esse mesmo produto interno}

Definindo um produto interno no espaço vetorial apresentado no enunciado:
    \[ \langle \begin{pmatrix} 
        a & b \\
        c & d \\
        e & f
       \end{pmatrix} ,\begin{pmatrix} 
        g & h \\
        i & j \\
        k & l
       \end{pmatrix} \rangle =  a \cdot  g + b \cdot h +
       c \cdot i + d \cdot j +
       e \cdot k + f  \cdot l 
        \]
Para que seja produto interno: 

\begin{enumerate}
    \item $ \langle u, v \rangle =  \langle v, u \rangle $
        
    \[ \langle \begin{pmatrix} 
        a & b \\
        c & d \\
        e & f
       \end{pmatrix} ,\begin{pmatrix} 
        g & h \\
        i & j \\
        k & l
       \end{pmatrix} \rangle =  a \cdot  g + b \cdot h +
       c \cdot i + d \cdot j +
       e \cdot k + f  \cdot l 
        \]
    
    \[ \langle \begin{pmatrix} 
        g & h \\
        i & j \\
        k & l
        \end{pmatrix},\begin{pmatrix} 
            a & b \\
            c & d \\
            e & f
       \end{pmatrix} \rangle =  g \cdot  a + h \cdot g +
       i \cdot c + j \cdot d +
       k \cdot e + l  \cdot f 
        \]
    Portanto por comutatividade da multiplicação de reais
    \[ \langle \begin{pmatrix} 
        g & h \\
        i & j \\
        k & l
        \end{pmatrix},\begin{pmatrix} 
            a & b \\
            c & d \\
            e & f
       \end{pmatrix} \rangle =  a \cdot  g + b \cdot h +
       c \cdot i + d \cdot j +
       e \cdot k + f  \cdot l 
        \]
    E assim:
    \[ \langle \begin{pmatrix} 
        g & h \\
        i & j \\
        k & l
        \end{pmatrix},\begin{pmatrix} 
            a & b \\
            c & d \\
            e & f
       \end{pmatrix} \rangle = \langle \begin{pmatrix} 
        a & b \\
        c & d \\
        e & f
       \end{pmatrix} ,\begin{pmatrix} 
        g & h \\
        i & j \\
        k & l
       \end{pmatrix} \rangle \]

    \item $ \langle u + v, w \rangle =  \langle u, w \rangle + \langle v, w \rangle $
    
    \[ \langle \begin{pmatrix} 
        a & b \\
        c & d \\
        e & f
       \end{pmatrix} + 
       \begin{pmatrix} 
        g & h \\
        i & j \\
        k & l
        \end{pmatrix},
        \begin{pmatrix} 
        m & n \\
        o & p \\
        q & r
        \end{pmatrix} \rangle = 
        \langle \begin{pmatrix} 
        a + g & b + h \\
        c + i & d + j \\
        e + k & f + l 
        \end{pmatrix} ,\begin{pmatrix} 
        m & n \\
        o & p \\
        q & r
        \end{pmatrix} \rangle \]
    Realizando o produto interno:
        \[\langle \begin{pmatrix} 
        a + g & b + h \\
        c + i & d + j \\
        e + k & f + l 
        \end{pmatrix} ,\begin{pmatrix} 
        m & n \\
        o & p \\
        q & r
        \end{pmatrix} \rangle 
        = (a + g) \cdot m + (b + h) \cdot n + (c + i) \cdot o + (d + j) \cdot p + (e + k) \cdot q + (f + l) \cdot r \]
    Por distributividade da multiplicação de reais:     
    \[\to am + gm + bn + hn + co + io + dp + jp + eq + kq + fr + lr \]
    Por associatividade da soma de reais:
    \[\to ( am + bn + co + dp + eq + fr) + (gm +  hn + io + jp + kq + lr)\]
    \[=
    \langle \begin{pmatrix} 
    a & b \\
    c & d \\
    e & f
    \end{pmatrix} ,\begin{pmatrix} 
    m & n \\
    o & p \\
    q & r
    \end{pmatrix} \rangle +
    \langle \begin{pmatrix} 
    g & h \\
    i & j \\
    k & l
    \end{pmatrix} ,\begin{pmatrix} 
    m & n \\
    o & p \\
    q & r
    \end{pmatrix} \rangle\]

    \item $\langle \alpha u, v \rangle = \alpha \langle u, v \rangle $
    
    \[ \langle \alpha \cdot \begin{pmatrix} 
        a & b \\
        c & d \\
        e & f
       \end{pmatrix} ,\begin{pmatrix} 
        g & h \\
        i & j \\
        k & l
       \end{pmatrix} \rangle
        = \langle \begin{pmatrix} 
        \alpha \cdot a & \alpha \cdot b \\
        \alpha \cdot c & \alpha \cdot d \\
        \alpha \cdot e & \alpha \cdot f
        \end{pmatrix} ,\begin{pmatrix} 
        g & h \\
        i & j \\
        k & l
        \end{pmatrix} \rangle
        \]
        Aplicando o produto interno:
        \[ \langle \begin{pmatrix} 
        \alpha \cdot a & \alpha \cdot b \\
        \alpha \cdot c & \alpha \cdot d \\
        \alpha \cdot e & \alpha \cdot f
        \end{pmatrix} ,\begin{pmatrix} 
        g & h \\
        i & j \\
        k & l
        \end{pmatrix} \rangle
        = \alpha \cdot a \cdot  g + \alpha \cdot b \cdot h +
        \alpha \cdot c \cdot i + \alpha \cdot d \cdot j +
        \alpha \cdot e \cdot k + \alpha \cdot f  \cdot l  \]
        Por distributividade de reais:
        \[ \to  \alpha \cdot (a \cdot g + b \cdot h + c \cdot i + d \cdot j + e \cdot k + f \cdot l) \]
        \[ = \alpha \cdot \langle  \begin{pmatrix} 
            a & b \\
            c & d \\
            e & f
           \end{pmatrix} ,\begin{pmatrix} 
            g & h \\
            i & j \\
            k & l
           \end{pmatrix} \rangle \]
    \item  $\langle u, u \rangle > 0 $ se $ u $ não é o vetor nulo
    \[\langle \alpha \cdot \begin{pmatrix} 
        a & b \\
        c & d \\
        e & f
       \end{pmatrix} ,\begin{pmatrix} 
        a & b \\
        c & d \\
        e & f
       \end{pmatrix} \rangle = a \cdot a + b \cdot b + c \cdot c + d \cdot d + e \cdot e + f \cdot f\]
    \[ = a^2 + b^2 + c^2 + d^2 + e^2 + f^2 \]
    Se $ a,\ b,\ c,\ d,\ e$ e $f$ são números reais não nulos, então obrigatoriamente serão positivos quando elevados ao quadrado.
\end{enumerate}

Definição de ortogonalidade de vetores:
    \[ \langle u, v \rangle = 0\]
Dois vetores ortogonais entre si no produto interno e espaço utilizado:
\[\mathcal{A} = \begin{pmatrix} 
    10 & 0 \\
    20 & 0 \\
    30 & 0
   \end{pmatrix} ,\mathcal{B} = \begin{pmatrix} 
    0 & 4 \\
    0 & 8 \\
    0 & 16
   \end{pmatrix} \]
Testando se são ortogonais, aplicando na definição estabelecida de produto interno:
\[ \langle \mathcal{A}, \mathcal{B} \rangle = 10 \cdot 0 + 0 \cdot 4 + 20 \cdot 0 + 0 \cdot 8 + 30 \cdot 0 + 0 \cdot 16 \]
Portanto:
\[ \langle \mathcal{A}, \mathcal{B} \rangle = 0 \]
\end{document}